\documentclass[12pt,a4paper]{article} 
\usepackage[T1]{fontenc}
\usepackage[francais]{babel}
\usepackage[latin1]{inputenc}
\usepackage{amsmath}
\usepackage{amssymb} %pour des symboles mathematique
\usepackage{graphicx} 
\usepackage{epstopdf} %pour utiliser les .eps
\usepackage{url}
\usepackage{fancyhdr}
\usepackage{fullpage}
\usepackage{array}
\usepackage[table]{xcolor} %pour les couleurs
\usepackage{multicol}
\usepackage[babel=true]{csquotes} %pour les citations

\usepackage{adjustbox} %pour resizer les tableaux

%%%% pour l'insertion de code %%%%%
\usepackage{listings}
\usepackage{color}

\definecolor{dkgreen}{rgb}{0,0.6,0}
\definecolor{gray}{rgb}{0,0,0}
\definecolor{mauve}{rgb}{0,0,0}

\lstset{emph={%  
    float, %
    },emphstyle={\color{gray}\bfseries}%
}%

\lstset{frame=tb,
  language=Matlab,
  aboveskip=3mm,
  belowskip=3mm,
  showstringspaces=false,
  columns=flexible,
  basicstyle={\small\ttfamily},
  numbers=none,
  numberstyle=\tiny\color{gray},
  keywordstyle=\color{black},
  commentstyle=\color{dkgreen},
  stringstyle=\color{black},
  identifierstyle=\color{mauve},
  breaklines=true,
  breakatwhitespace=true
  tabsize=3
}

%%%%%%%%%%%%%%%%%%%%%%

%\pagestyle{myheadings}
%\markright{Tableau blanc interactif}

\title{\LARGE \textbf{Project - The Game Cube\textregistered}\\
	\bigskip
	\bigskip
	\large Technologies for human-computer interactions}
\author{Aina Rasolonjatovo Alain Nary Andriambelo \\ Samuel Constantino}
\date{Spring 2014}

\parindent=0cm
\parskip=6pt

\begin{document}
	\maketitle

%%%%%%%%%%%%%%%%%%%%%%%

\section{Introduction}

The goal of this project is to build a game exploring original human-machine interactions. For that, we used a galvanic skin response (GSR) sensor as the main mean of interaction and the \texttt{Blender} 3d software to create the game. In our project, we use the GSR data - which measures skin conductance - as an indicator for stress and arousal. The goal of the game is for the player to manage her stress level and relax. 

\section{Project}

\subsection{Sensor data}

We started from the premise that the GSR sensor could be used as an indicator of excitement. Psychological arousal is linked with nervous activity that changes skin conductance. However, data from the GSR sensor aren't necessarily stable and skin conductance is subject to sudden changes (spikes) that doesn't reflect a general state of being.

In our game, we used a statistical analysis technique called \textbf{moving average} to smooth out the data. The advantage of the moving average is that it reduces the impact of spikes but still preserves long-term tendencies. This technique is usually used in finance to represent 
time 
disadvantage retard

example

formula
why use
	constant
	no peak
	directly translated into rotation
	comparison with....

\subsection{Game design}

missions etc

\subsection{Blender}

style

\section{Evaluation}



\section{Conclusion}

\end{document}